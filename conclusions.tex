\section{Conclusions}

Despite the ubiquity of Informatics, in any area we examined there were a significant minority of surveyed universities that have not really engaged with interdisciplinarity.  This does not preclude individual academics within these universities working on multi-discipline research and teaching. On the other-hand there are Informatics academics who are concerned that the pressure towards multi-disciplinary research is at the cost of core Informatics research. How much a university's leadership want to encourage interdisciplinarity can be seen in its policies and financial support for staff and centres.  The range is very large from no policy or financial support to using significant resources for hiring staff and setting up and funding centres. The most commonly found centres are in Data Science and this is an arena which is largely seen to arise from Informatics and Statistics. It seems quite early to see a pattern on how universities are going to develop with respect to interdisciplinary research. As to joint teaching, there are a very wide range of courses offered that include Informatics. 

\subsection*{Acknowledgements} We would like to thank all the respondents who wrote many thoughtful answers and provided the raw data. In addition to the authors, there are many people who have put significant time into both designing the questions and reading earlier drafts. These include Stuart Anderson, Luis Caires, Brian Keegan, Hannes Werthner, and Ulf Lesser.

Finally, this document would not exist without the broad and extensive help from Svetlana Tikhonenko of the Informatics Europe office. 
