\section*{Executive Summary}
In this report, we  discuss the results of the online survey conducted by Informatics Europe Working Group on the Wide Role of Informatics at Universities. The main goals were to understand the value universities place on interdisciplinary research and teaching, what happens in practice with hiring and supporting interdisciplinary academics, and what structures are in place to support interdisciplinary work. We also examined Data Science's impact in detail, given its rapid rise and importance. Forty eight universities from nineteen European countries have participated in the survey providing answers on these strategic topics. 

The results of our investigation have shown that: 
\begin{itemize}
\item In any area examined a significant majority of surveyed universities were engaged with interdisciplinarity. However, there were Informatics academics concerned about the development of interdisciplinary research mainly owing to limited funding, low esteem compared with discipline-specific research or lack of strategic direction.
\item The majority of surveyed universities run joint degrees, including Informatics, most frequently in the area of Business and Economics, Natural and Life Sciences, and Engineering. The universities not already offering joint degrees showed a considerable interest in running new joint degrees including Informatics in Mathematics and Statistics, Natural and Life Sciences, Law, Social and Political Sciences, and Business and Economics.
\item With regard to teaching of Informatics in non-informatics programmes, there was not a uniform pattern. While for some universities there existed a clear discipline-responsibility, in others there was no clear policy about which department teaches Informatics in non-informatics programmes. Moreover, in several universities the lack of human resources prevented the Informatics departments from being in charge of teaching Informatics subjects in non-informatics degree programmes.
\item In terms of policies for interdisciplinarity and financial support for staff and centres, the range of answers was very large from no policy or financial support to using significant resources for hiring staff and setting up and funding centres. In the cases where universities were largely autonomous from national agencies, hiring interdisciplinary researchers was encouraged when there was some funding, often by third parties, dedicated to this. Respondents from countries where the hiring system was strongly regulated by some national agency highlighted the difficulty to introduce some flexibility and to define long-term plans which include multidisciplinarity.
\item The most commonly found centres were in Data Science, an area which was largely seen to emerge from Informatics and Statistics. According to the majority of surveyed universities, the rise of Data Science has changed the perception of Informatics resulting in increasing relevance of ethics and other social aspects and in developing introductory courses on digital literacy and skills in all study programs. Informatics was considered to be the main knowledge centre in the digital transformation of society and many initiatives are under way changing how Informatics is perceived.
\end{itemize}

For all the questions there were a significant number of universities that have not engaged jn official interdisciplinary activity.


