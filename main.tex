\documentclass[11pt]{amsart}
\usepackage{geometry}                % See geometry.pdf to learn the layout options. There are lots.
\geometry{a4paper}                   % ... or a4paper or a5paper or ... 
%\geometry{landscape}                % Activate for for rotated page geometry
\usepackage[parfill]{parskip}    % Activate to begin paragraphs with an empty line rather than an indent
\usepackage{graphicx}
\usepackage{amssymb}
\usepackage{epstopdf}
\DeclareGraphicsRule{.tif}{png}{.png}{`convert #1 `dirname #1`/`basename #1 .tif`.png}

\title{The Wide Role of Informatics at Universities}
\author{The Author}
%\date{}                                           % Activate to display a given date or no date

\begin{document}
\maketitle
\section{Introduction}

In the 1970s with the advent of the personal computer we entered into the Digital or Information Age. However it has only been in this century with the ubiquity of the internet, the smartphone, and the internet of things that digital has become truly pervasive. How do universities respond to this massive change? Informatics Europe established in 2018 a new working group to investigate what universities are doing to ensure that non-informatics teaching and research is informed by best practice in Informatics.

To better understand the state of affairs on this topic and discover best practices at European Universities, the working group conducted an online survey. We invited heads and members of Informatics/Computer Science/IT Departments (Schools, Faculties, Institutes) to complete a questionnaire. The request to fill out our survey was sent to all Informatics Europe members and it was also publicly available from the Informatics Europe website. The questionnaire was filled out autumn 2018. Forty eight universities (almost all IE members) from eighteen countries filled it out. The list of participants is in table~\ref{tab:unis}.

Our survey was wide ranging. We wanted to understand how universities valued interdisciplinary research, about teaching Informatics to non-specialist students, what happens i n practice with hiring and supporting interdisciplinary academics. We wanted to know about how Data Science in particular fits into universities and finally the structures in place to support interdisciplinary work. The survey questions are in Appendix~\ref{appendix}. 

How Informatics (also called Computer Science or Computing) should position itself in a university is a political decision. The extremes we could imagine range from primarily being a service department to primarily being a research area that is isolated from other departments.


\begin{table}
\begin{center}
\label{tab:unis}
\begin{tabular}  {|r|l|l|}
\hline
&{\bf Country} & {\bf University}\\
\hline
1.&Austria   &  TU Wien\\
\hline
2.&Belgium   & Universit\' e Catholique de Louvain \\
\hline
3.&Bulgaria  &  Sofia University St. Kliment Ohridski\\
\hline
4.&Czech Republic & Masaryk University\\
\hline
5.&Denmark  &  Aalborg University\\
& &  IT University of Copenhagen \\
&  &  University of Southern Denmark\\
\hline
6.&Estonia   & Tartu University \\
\hline
&Finland  & Aalto University \\
\hline
7.&Germany  & RWTH Aachen \\
&  & Humboldt-Universit\H at zu Berlin \\
&  &  Paderborn University\\
&  & University of Stuttgart \\
\hline
8.&Hungary   &   E\" otv\"os Loránd University\\
\hline
9.&Ireland  &  Technological University Dublin\\
\hline
10.&Italy   & University of Bari Aldo Moro\\
 &  &  Universit\` a di Torino\\
 & & Alma Mater Studiorum - Università di Bologna \\
 & & Universit\` a degli Studi di Milano \\
&  & Politecnico di Milano \\
& & Universit\` a Roma Tre \\
&  &  Universit\` a degli Studi di Milano-Bicocca\\
&  & Universit\` a degli studi ``G. d' Annunzio'' Chieti Pescara \\
\hline
11.&Latvia   & University of Latvia\\
&  & Transport and Telecommunication University \\
\hline
12.&Netherlands  &  Delft University of Technology\\
& & Tilburg University \\   
& &  Utrecht University \\
 \hline
13.& Portugal  & Universidade Nova de Lisboa \\
 \hline
14.&Romania  & Babes-Bolyai Univ. Cluj-Napoca \\
\hline
15.&Spain  & University of Almeria \\
&  & Universitat Politecnica de Catalunya \\
&  &  University of Extremadura\\
&  &  University Jaume I\\
&    & University of M\' alaga \\
&  &  Complutense University of Madrid\\
&  &  University Oviedo\\
&  & Universidad de Valladolid \\
\hline
16.& Sweden   &  Chalmers | Gothenburg University\\
\hline
17.&Switzerland   & University of Bern\\
& & EPFL \\
&  &  University of Lugano \\
 &    & ETH Z\" urich \\
&   &  University of Z\" urich\\
\hline
18.&UK   &  Cambridge University\\
&   &  University of Edinburgh\\
 &  & Imperial College London\\
&  & University of Oxford \\
\hline
\end{tabular}
\caption{Participating Universities}
\end{center}
\end{table}



















%\subsection{}

\section{research}

Universities are normally structured into disciplines which foster disciplinary research. However, the ubiquity of Informatics in our culture has led to pressures for research that is interdisciplinary. Pressures in favour of such research comes from academics themselves, student interests, external funding sources, and sometimes from university leadership. The 
following subsections discuss the answers obtained for each specific
question.

\subsection{Desirability of interdisciplinary research}

\begin{figure}[h]
\centering
\includegraphics[width = \linewidth]{charts/1a.png}
\caption{What is the University attitude towards Interdisciplinary research?}
\label{sect1:Uattitude}
\end{figure}

The first part of the survey questioned respondents on university attitudes and actions in respect of interdisciplinary research.\footnote{The survey does not differentiate between interdisciplinary work in general and that with an Informatics component. Given who answered the questionnaire, one can assume that Informatics is included} A large majority (71\%) claimed that their university encouraged interdisciplinary research when compared with single discipline research (see Figure~\ref{sect1:Uattitude}). This seems to imply that universities favour interdisciplinary research over single discipline research.  However, several respondents indicated that their encouragement was largely `theoretical' and accompanied by little, if any, funding. Some respondents said that much of the interdisciplinary work at their institution occurred between departments other than Informatics. Only one respondent indicated that their university actually discouraged interdisciplinary research although others mentioned that their departments were judged, usually nationally, against discipline-specific criteria.

\subsection{Department attitude towards Interdisciplinary research}


\begin{figure}[h]
\centering
\includegraphics[width = \linewidth]{charts/1b.png}
\caption{What is the Department attitude towards interdisciplinary research?}
\label{sect1:Dattitude}
\end{figure}

With the same question directed at Informatics Departments rather than the whole university (see Figure~\ref{sect1:Dattitude}), two thirds of respondents still claimed that interdisciplinary research was favoured over single discipline topics. However, similar comments are made about encouragement being in principle rather than in practice and about being judged on discipline-specific criteria.

\subsection{University support}

\begin{figure}[h]
\centering
\includegraphics[width = \linewidth]{charts/1c.png}
\caption{Are there interdisciplinary areas of research where your university
could enter but aren't due to lack of university support?}
\label{sect1:support}
\end{figure}

However, just over half (51\%) of the respondents recorded (see Figure~\ref{sect1:support}) that their university supported all areas of interdisciplinary research which required  support. Others (30\%) mentioned a variety of potential informatics areas where university support for interdisciplinary research was lacking. Others talked of the need for strategic planning to direct interdisciplinary efforts or of the need to focus given the wide range of potential opportunities.

\subsection {Additional support}

\begin{figure}[h]
\centering
\includegraphics[width = \linewidth]{charts/1d.png}
\caption
{Are there other players who have helped increase the
interdisciplinary research in your university?}
\label{sect1:additional}
\end{figure}

When asked about external support for interdisciplinary research directed towards their university (see Figure~\ref{sect1:additional}), 50\% of the respondents responded positively with 40\% stating that national public funding sources had helped to increase interdisciplinary research. A further 22\% mainly discussed specific formal or informal arrangements between their department and others in their institution.

\subsection{Final thoughts}

Respondents were asked to make some more general comments. Not all respondents were especially supportive of interdisciplinary research per se. It was noted that, because some funding streams demand interdisciplinarity, it is possible that `artificial collaborations' were formed that attracted the funds but did not make good use of the capabilities of the researchers. Frequently interdisciplinary projects are focussed on how information technology can serve the other discipline so the progress made and any breakthroughs that occur advance the other discipline but have no impact on the development of Informatics. One respondent suggested that the excitement and interest in supporting interdisciplinary projects could make it likely that lower quality proposal were accepted (compared with single discipline ones).

Respondents with more positive attitudes towards interdisciplinary research were often nevertheless concerned about its development mainly owing to limited funding, low esteem compared with discipline-specific research or lack of strategic direction.


\section{teaching}

When teaching is run by departments it is easier to have single discipline degrees rather than joint degrees, and there is no shortage of students wanting to study Informatics as a single discipline. Nonetheless there is pressure (from perspective students, academics, industry, and sometimes university leadership) to have joint degrees. The 
following subsections discuss the answers obtained for each specific
question.

\subsection {Joint degrees}

\begin{figure}[h]
\includegraphics[width = \linewidth]{charts/2a.jpg}
\caption{Does your university run joint degrees?}
\label{sect3:joint}
\end{figure}

30\% of the universities do not run a joint degree that includes Informatics (see Figure~\ref{sect3:joint}). Within this group of universities, some specified that all their programs entail technical aspects of IT, such as programming or data base technology.  At some of these universities there are plans for some joint programmes, e.g. a Data Science BSc programme that joins CS, Maths and Industrial Engineering, and an MSc in Game Design and Production jointly with the Arts School, but these are collaborative initiatives in new directions, where the CS Department is one of the partners or the Business School has their own small Informatics programme for the new degree.

The remaining 70\% of the universities run joint degrees, the most popular joint degrees including Informatics are  Business and Economics (Business Informatics; CS and Business; Computing and Economics; Information systems combining Informatics and Business Administration; CS and Management; Informatics and Economics; Informatics and Finance; Economics and Business Informatics; Data Science and Entrepreneurship) followed by Mathematics and Statistics (Informatics and Mathematics; Data Science; Informatics and Applied Mathematics; Informatics and Statistics), Natural and Life Sciences (Bioinformatics; Informatics and Natural Sciences; CS and Physics; AI for Biomedicine; Precision Medicine; Geoinformatics; Chemistry and Informatics; Biology and Informatics; Informatics Health) and Engineering (Computational Engineering; Computer Engineering; Electronics and Information Engineering; Informatics and Electronics; Informatics and Telecommunications; Informatics and Cybernetics; Informatics and Mechatronics; Informatics and Aerospace Engineering; Informatics and Civil Engineering; Informatics and Industrial Engineering). Joint degrees in Informatics plus Arts, Design and Media (Technical Communication; Design Informatics; CS and communication, CS and design; ICT and media; Informatics and information science; Informatics and library science) or Law, Political and Social Sciences (Law and Informatics; Social sciences and Informatics; Data mining for political sciences; Informatics and Psychology; Data Science and society; Cognitive Science and AI) are not very frequent at the consulted universities, they represent only the 11\% of the cases. Appendix~\ref{apx:teaching} summarizes  the joint degrees (BSc. and MSc) offered by one or more universities and the countries where they are located.

\subsection{Plans for changes in joint degrees} 

\begin{figure}[h]
\includegraphics[width = \linewidth]{charts/2b.jpg}
\caption{Are there plans to run new joint degrees or to close down joint degrees?}
\label{sect3:change}
\end{figure}

In general, the situation is  quite stable for those universities that are currently offering joint degrees (see Figure~\ref{sect3:change}). Most of the universities not already offering joint degrees show a significative interest in running new joint degrees. The most popular joint degrees to be run in the future are in the subject of Mathematics and Statistics for which at least eight universities have shown interest, followed by the subject of 
Natural and Life Sciences  and 
Law, Social and Political Sciences and finally the area of  Business and Economics. 

\subsection{Teachers for external departments}
\begin{figure}[h]
\includegraphics[width = \linewidth]{charts/2c.jpg}
\caption{Who teaches the Informatics component of non-Informatics degrees?}
\label{sect3:teachers}
\end{figure}

The results of the survey indicate that half of the universities (50\%) give the responsibility of teaching informatics subjects to non-informatics degree students to members of the Informatics department (see Figure~\ref{sect3:teachers}). In an additional 21\% of the universities, the responsibility of teaching Informatics is shared among the Informatics department and other departments involved in the joint degree; some of the universities specify that only the general/basic Informatics subjects of non-Informatics degrees are taught by academics in the Informatics department (for example programming) but when the subject is related to any particular contents of the degree and the Informatics, then the subject is taught by the teachers with profile related with the specific degree. For example, the Bioinformatics of the Biotechnology degree is taught by Chemists. In other universities, Informatics component of non-informatics degree programmes is sometimes taught by the Informatics department, especially the more advanced levels. Some of the Informatics departments have not enough human resources to acquire teaching responsibilities  for non-Informatics degrees . A significative percentage of the universities consulted (29\%) recognize that Informatics components of joint degrees are taught by other departments such as Physics, Mathematics, Economics, etc., depending on the subject of the joint degree.

\subsection{Training of Informatics teachers outside of an Informatics department}

\begin{figure}[h]
\includegraphics[width = \linewidth]{charts/2d.jpg}
\caption{What training do teachers of Informatics outside of the Informatics department have?}
\label{sect3:whoteaches}
\end{figure}
27\% of the respondents reported that all Informatics taught in their university was taught by members of the Informatics department (see Figure~\ref{sect3:whoteaches}).
Additionally, 22\% of the answers specify that Informatics is taught by Computer Scientists. 
Most of the universities participating in the survey recognize that some of the people who teach Informatics for students of non-informatics degree do not have a background in Computer Science (51\%). Usually, when the Informatics subjects are  taught by non Computer Scientists, the teachers have a background formation in the same degree the students are following; e.g. electrical engineers in the Electrical Engineering Schools, Economics/Management people at the Business School, Physicists or Engineers in Robotics or Industrial Engineering degrees. Additionally, in some universities the basic Informatics courses  are taught by non Computer Scientists.   


\subsection{Final thoughts}

The range of the answers is really broad. For some universities there exists a clear discipline-responsibility, but in others there are no clear policy about which department teaches Informatics in non-informatics programmes; in several universities the lack of human resources prevents the Informatics departments from being in charge of teaching Informatics subjects in non-informatics degree programmes.

\section{people}

If interdisciplinary research and teaching are to thrive, in addition to a positive hiring policy there needs to be good career development for those that undertake it.
 In general, it is possible to affirm that the situation,
even if significantly different from case to case, reveals a
significant level of immaturity that will have to be overcome in the
near future if interdisciplinary research and teaching are to thrive. The good news is that some universities, even if in
a non-completely structured way, are investing significant effort to
increase the presence of interdisciplinary faculty among research and
teaching staff. More time is certainly needed to assess the
effects of these investments and to see a change in the most
conservative countries in Europe. The 
following subsections discuss the answers obtained for each specific
question.

\subsection{Interdisciplinary hiring}\label{sec:hiring}

\begin{figure}[h]
\centering
\includegraphics[width = \linewidth]{charts/3a.png}
\caption{Does your university explicitly hire academics
  who focus on interdisciplinary research?}
\label{sect3:hirings}
\end{figure}

63\% of the respondents have affirmed that their university does not
explicitly hire interdisciplinary researchers (see Figure~\ref{sect3:hirings}). In Italy this is 
due to the organization of research areas in distinct \emph{scientific
  sectors}, which are mostly related to a single discipline and cannot be easily revised to follow the advances of
research and technology. Spain appears to show similar problems.

Among the 37\% of positive respondents, some identify bioinformatics
as one of the areas where multidisciplinary researchers are hired. 
Other identified areas concern man-machine interaction, medical
informatics, AI/data science, and media informatics/game design.

\subsection{Affiliation of interdisciplinary faculties}
\begin{figure}[h]
\centering
\includegraphics[width = \linewidth]{charts/3b.png}
\caption{Are faculty rooted in a department, have a joint appointment across departments, or rooted in a centre?}
\label{sect3:affiliation}
\end{figure}

In 74\% of the cases, multidisciplinary researchers are rooted within
a department (see Figure~\ref{sect3:affiliation}). 
According to the comments associated to this question,
this seems to be due to the need to assign every faculty to a specific
department. The respondents, however, note that such researchers spend
also part of their time in a multidisciplinary centre or in another
department. 

\subsection{Assessment of interdisciplinary faculties}


\begin{figure}[h]
\centering
\includegraphics[width = \linewidth]{charts/3c.png}
\caption{How is their quality judged for both appointment and for promotion?}
\label{sect3:assessment}
\end{figure}

As shown in Figure~\ref{sect3:assessment}, there is an equal distribution between universities
where the appointment/promotion assessment is performed at the
department level and universities where this happens across
departments. Analysing the specific comments by the respondents, it is
difficult to find common patterns as the mechanisms for appointing and
promoting faculties appear to vary significantly from country to
country. 

\subsection{Planned initiatives concerning multidisciplinary hirings}

\begin{figure}[h]
\centering
\includegraphics[width = \linewidth]{charts/3d.png}
\caption{Are there any initiatives planned to hire in interdisciplinary areas?}
\label{sect3:planned}
\end{figure}

As shown in Figure~\ref{sect3:planned}, the answer to this question appear to be quite similar to the ones
discussed in Section~\ref{sec:hiring}. Also in this case, 63\%of
respondents do not see any plan to hire multidisciplinary researchers
while among those who see these plans in place natural life and
science and, in particular, bioinformatics, appear to be the most
targeted field. 

\subsection{Final thoughts}

The answers to this question show that the situation is still quite
immature. In the cases where universities are largely autonomous from
national agencies, hiring interdisciplinary researchers is
encouraged when there is some funding, often by third parties, dedicated to
this. Even in this case, respondents highlight the difficulty of comparing
researchers with different background and skills and the current
lack of complete understanding of the phenomenon given the limited
number of multidisciplinary researchers that are currently in the
system. 

Respondents from countries where the hiring system is strongly
regulated by some national agency, highlight the difficulty to
introduce some flexibility and to define long-term plans
which include multidisciplinarity as an important aspect. 

\section{Data Science}

Eduard Groller
\newpage
\section{structure}

Susan Eisenbach


\subsection{Interdisciplinary centres}

\begin{figure}[h]
\centering
\includegraphics[width = \linewidth]{charts/5a.jpg}
\caption{What are the interdisciplinary centres?}
\label{sect5:centres}
\end{figure}

28\% of respondents say their university does not have real interdisciplinary centres (see Figure~\ref{sect5:centres}). Of those who commented on why the lack of centres only Aalto University actually replied that their management was averse to setting up additional administrative structures. The rest just said there were informal groupings, but nothing officially supported. 46\% of all of the interdisciplinary centres are set up primarily for research and only 18\% for teaching. The rest are primarily involved with industry.

There are a broad range of centres in the different universities -- clearly what expertise is in a university and what the structure of the different departments/schools/faculties impacts which centres are set up in addition to the existing primary structures. The most common centres mentioned with a significant Informatics component are in Computational Science (Delft, Aachen, Southern Denmark, Catalunya, Aalborg), Data Science (Imperial College, P Milano, Lugano, Paderborn, Tilburg),  Life Science (Babes-Bolyai,  Edinburgh, Humboldt, Lugano, Masaryk, Tarfu), Digital Society (ETH, Zurich, Sofia), Energy (Delft, ETH, TU Wien), and Security(Edinburgh, Imperial, P Milano).   There were two universities with the following centres: Biomedical Engineering (EPFL, Catalunya), Environment/Climate (ETH, Humboldt),  Medical Imaging (ETH, Imperial), and  Complex Systems (TU Wien, Utrecht). There are a wide range of centres which only mentioned at one university: Health (Delft),  FinTech (Zurich), Digital Humanities (E\" otv\"os Lor\'and), Robotic Surgery (Imperial ), Cognitive Ageing (Edinburgh), Bioinformatics (P Milano), and Geoinformatics (P Milano), 
 


\subsection{Purpose of interdisiciplinary centres}

\begin{figure}[h]
\centering
\includegraphics[width = \linewidth]{charts/5b.jpg}
\caption{Why were the centres created?}
\label{sect5:reasons}
\end{figure}

45\% of all of the interdisciplinary centres are set up primarily for research and only 18\% for teaching (see Figure~\ref{sect5:reasons}). The rest are primarily involved with industry collaboration or consultancy.

\subsection{ Ownership of interdisciplinary centres}

\begin{figure}[h]
\centering
\includegraphics[width = \linewidth]{charts/5c.jpg}
\caption{Which entity control the interdisciplinary centres?}
\label{sect5:owners}
\end{figure}

Of the 36 respondents, 21 (or 58\%) are independent entities within their university, 12 (or 1/3) are co-owned by the departments that are involved and the rest have a single department that owns them (see Figure~\ref{sect5:owners}). It is surprising that so many are separate entities as this means if they are not self-funding money will be an issue.

\subsection{ Location of interdisciplinary centres}

\begin{figure}[h]
\centering
\includegraphics[width = \linewidth]{charts/5d.jpg}
\caption{ Where are the centres located?}
\label{sect5:locations}
\end{figure}

(see Figure~\ref{sect5:locations})

\subsection{Funding of interdisciplinary centres}

\begin{figure}[h]
\centering
\includegraphics[width = \linewidth]{charts/5e.jpg}
\caption{Who funds interdisciplinary centres?}
\label{sect5:funding}
\end{figure}

Only 25\% of the interdisciplinary centres reported on are funded entirely externally, the funding of the rest being equally split between entirely internal and mixed sources of funding (see Figure~\ref{sect5:funding}). In the majority of cases where funding is entirely internal, the bulk of the actual cash seems to come from central funds with departments providing resources `in kind'. Frequently, time-limits are expressed (five and six years are mentioned) after which the centre is expected to be self-financing. For the universities that reported on (entirely or partially) external funding, in many cases only government and EU programmes were explicitly cited as sources of funds.

\subsection{Planning for changing interdisciplinary centres}

\begin{figure}[h]
\centering
\includegraphics[width = \linewidth]{charts/5f.jpg}
\caption{Are there changes planned for setting up or closing centres?}
\label{sect5:changes}
\end{figure}

A quarter of respondents report on plans to set up new centres (see Figure~\ref{sect5:changes}). Some describe a notion of continuous evolution of interdisciplinary work. Only AI was explicitly mentioned as a target for the development of new centres. Other respondents, although not explicitly planning a new centre, mention the issue of the periodic review of existing centres citing various options including merging centres and/or creating new centres.  

\subsection{Drivers for new activities}

\begin{figure}[h]
\centering
\includegraphics[width = \linewidth]{charts/5g.jpg}
\caption{What are the drivers for new centres?}
\label{sect3:drivers}
\end{figure}
                                                                                                                    
Nearly one third of respondents reported on internal drivers and pressures bearing on innovative activity (see Figure~\ref{sect5:drivers}). Amongst the drivers, academic curiosity of staff and students was cited alongside a need for research collaboration. Pressures included demands to increase students enrolment, to modify the curriculum and university initiatives to set up a centre. One university also mentioned limitations of student numbers and limitations on joint degrees that inhibited their development goals.

The other respondents addressed external drivers and pressures. The most significant cited pressure concerned the societal influence of globalisation together with an associated driver on universties to promote innovation and  technology transfer (47\%).  The next most significant pressure is the search for funding driven by government initiatives (30\%) whilst other respondents observed the expanding role of Informatics in other disciplines and the pressure on Informatics departments to support these disciplines (20\%). Finally, one respondent mentioned competition between universities as an external pressure.

\subsection{Support for interdisciplinary work}

\begin{figure}[h]
\centering
\includegraphics[width = \linewidth]{charts/5h.jpg}
\caption{How much support is provided for interdisciplinary work?}
\label{sect5:support}
\end{figure}

Respondents were evenly split over this question (see Figure~\ref{sect5:support}) although several of those who claimed institutional support were rather equivocal - "I would guess so" and ``Some departments \ldots ''. Respondents who reported no institutional support divided into those who stipulated some form of external support and those who did it ``as a hobby'' (~25\%).

\subsection{Strategic vision }
\begin{figure}[h]
\centering
\includegraphics[width = \linewidth]{charts/5i.jpg}
\caption{Interdisciplinary hirings}
\label{sect5:strategy}
\end{figure}

More than half of the respondents reported on centres created from strategic initiatives (see Figure~\ref{sect5:strategy}). Many of these were oriented towards Informatics themes (FinTech, Crypto-currencies, Data Science) but several other types of centre were mentioned (Learning and Education, Cultural Heritage, Sustainability and Energy).

\subsection{Official strategic vision}
\begin{figure}[h]
\centering
\includegraphics[width = \linewidth]{charts/5i.jpg}
\caption{Is there an official strategy to widen the role of Informatics?}
\label{sect5:official}
\end{figure}

Respondents were exactly split on this question (see Figure~\ref{sect5:official}). Of those who answered positively, the emphasis was on multidisciplinarity for about half the respondents. Informatics topics cited by others included Cyber Security, Data-driven Innovation, Intelligent Systems, Applied Computer Science and Digital Humanities. Respondents who answered ``No'' were not very forthcoming with their comments.

\subsection{Final thoughts}

Nineteen respondents contributed their overall views on the current situation in their universities. One response was wholeheartedly supportive citing good funding, strong collaboration and a sound international reputation as attractive to world-class researchers. Other commentators mentioned limited or non-existent funding and other, higher priorities (like increased student enrolment) as factors which retarded interdisciplinary inttiatives. Two universities thought that Informatics was too junior a partner in the context of their university to make much impact.

By far the most significant issue concerned the nature of either the central or departmental strategic direction. Three respondents asked for greater freedom for individual researchers to be more creative with ideas, contacts and funding.  However, there were ten contributors who asked for better communication between faculties, more structured research management or further internationalisation. A few just wanted more substance to the strategy - ``It is only a goal without supporting instruments. '';  ``Still under construction - too early to conclude \ldots ''.





\newpage
\appendix
\section{Survey: The Wide Role of Informatics at Universities}\label{apx:survey}
\begin{enumerate}
\item Research
\begin{enumerate}
\item When compared with single disciplinary research, does your
university encourage or discourage (or neither) interdisciplinary
research? If so how? (e.g. funding, time, physical centres)
\begin{itemize}
\item Encourage
\item Discourage
\item Neither encourage nor discourage
\end{itemize}
\item Does your Informatics department encourage or discourage (or
neither) interdisciplinary research? If so how?
\begin{itemize}
\item Encourage
\item Discourage
\item Neither encourage nor discourage
\end{itemize}
\item Are there interdisciplinary areas of research where your university
could (should) enter but aren't due to lack of university support? If so
what are they?
\item Are there other players who have helped increase the
interdisciplinary research in your university?
For example has a funding body focused a programme on
interdisciplinary PhD studentships which academics applied for?If so
what external organisations and what programmes have increased
interdisciplinary research at your university?
\item Please comment on any advantages or disadvantages you perceive
of your university's arrangements.
\end{enumerate}
\item Teaching
\begin{enumerate}
\item Does your university run joint degrees (e.g. X and Informatics, Informatics and X, X with Informatics, Informatics with X). If yes,
what are they?
\begin{itemize}
\item Yes
\item No
\end{itemize}
\item Are there plans to run new joint degrees or to close down joint
degrees? If yes what are they?
\begin{itemize}
\item Run new joint degrees
\item Close down joint degrees
\item Neither run nor close down
\end{itemize}
\item Who teaches the Informatics component of non-Informatics degrees? For example, is programming taught to Physicists by members
of the Physics department, of the Informatics department or is there a
servicing organisation within your university that teaches Physics
students to code (or some other mechanism)?
\item If Informatics is taught by people not located in an Informatics
 department are they Computer Scientists by training or research?
 \begin{itemize}
\item They are Computer Scientists
\item They are not Computer Scientists
\item Informatics is not taught by people not located in an Informatics department
\end{itemize}
\item Please comment on any advantages or disadvantages you perceive of your university?s arrangements.
\end{enumerate}
\item People
\begin{enumerate}
\item Does your university explicitly advertise/hire academics who focus
on interdisciplinary research?
\begin{itemize}
\item Yes
\item No
\end{itemize}
\item  Are they rooted in a department, have a joint appointment across
departments, or rooted in a centre?
\begin{itemize}
\item Rooted in a department
\item Have a joint appointment across departments
\item Rooted in a centre
\end{itemize}
\item How is their quality judged for both appointment and for promotion?For example are they judged according to the criteria of one
of the departments or both? Are the people who judge from a single
department or both?
\item Are there any initiatives planned to hire in interdisciplinary areas?
\begin{itemize}
\item Yes
\item No
\end{itemize}
\item Please comment on any advantages or disadvantages you perceive of your university?s arrangements.
\end{enumerate}
\item Data Science
\begin{enumerate}
\item Which department in your university is seen to own this area? Is it
Informatics, Statistics, jointly or somewhere else?
\begin{itemize}
\item Informatics Department
\item Statistics Department
\item Jointly Informatics and Statistics Department
\item Somewhere else (please specify)
\end{itemize}
\item Has the rise of this area changed the perception of Informatics
overall in your university?
\begin{itemize}
\item Yes
\item No
\end{itemize}
\item Please comment on any advantages or disadvantages you perceive of
your university?s arrangements.
\end{enumerate}
\item Structure
\begin{enumerate}
\item Does your university set up centres for interdisciplinary work? If
yes can you say which they are?
\begin{itemize}
\item Yes
\item No
\end{itemize}
\item Are they for research, translation (technology transfer),
consultancy, and/or teaching?
\begin{itemize}
\item Research
\item Translation (technology transfer)
\item Consultancy
\item Teaching
\end{itemize}
\item Are they rooted in a single department (say which one), owned by
the departments involved or independent?
\begin{itemize}
\item Rooted in a single department
\item Owned by the departments involved
\item Independent
\end{itemize}
\item Are they physically located within a department, nearby or
elsewhere on campus?
\begin{itemize}
\item Within a department
\item Nearby a department
\item Elsewhere on campus
\end{itemize}
\item How are any centres funded? Does the university provide any
money to startup or are they funded by external money? Does the
university provide longer term money?
\item Are there plans to set up more centres or to close centres? If so
what will they be?
\begin{itemize}
\item Set up more centres
\item Close centres
\item Neither set up nor close
\end{itemize}
\item What are the drivers or pressures (both internal to the department/
school/faculty/university and external to the university)
that you see on the horizon that may lead to new activity?
\item Is substantial interdisciplinary work undertaken by academics
without any institutional or department support?
\begin{itemize}
\item Without any institutional or department support
\item With an institutional or department support
\end{itemize}
\item Are there any centres for interdisciplinary work that have been set
up due to a strategic decision by the university or
department/school/faculty rather than as supporting activities of
existing faculty? If so which centres?
\item Does your university have something in their official strategy to
widen the role of Informatics or to encourage interdisciplinary
research? If so what is it?
\item Please comment on any advantages or disadvantages you perceive of your university?s arrangements.
\item Is there anything we have missed in the survey that you wish to tell us?
\end{enumerate}
\end{enumerate}
\newpage
\section{The participants}\label{apx:names}
{\tiny{
\begin{table}[h]
\begin{center}
\begin{tabular}  {|r|l|l|}
\hline
&{\bf Country} & {\bf University}\\
\hline
1.&Austria   &  TU Wien\\
\hline
2.&Belgium   & Universit\' e Catholique de Louvain \\
\hline
3.&Bulgaria  &  Sofia University St. Kliment Ohridski\\
\hline
4.&Czech Republic & Masaryk University\\
\hline
5.&Denmark  &  Aalborg University\\
& &  IT University of Copenhagen \\
&  &  University of Southern Denmark\\
\hline
6.&Estonia   & Tartu University \\
\hline
7.&Finland  & Aalto University \\
\hline
8.&Germany  & RWTH Aachen \\
&  & Humboldt-Universit\H at zu Berlin \\
&  &  Paderborn University\\
&  & University of Stuttgart \\
\hline
9.&Hungary   &   E\" otv\"os Lor\'and University\\
\hline
10.&Ireland  &  Technological University Dublin\\
\hline
11.&Italy   & University of Bari Aldo Moro\\
 &  &  Universit\` a di Torino\\
 & & Alma Mater Studiorum - Università di Bologna \\
 & & *Universit\` a degli Studi di Milano \\
&  & Politecnico di Milano \\
& & Universit\` a Roma Tre \\
&  &  Universit\` a degli Studi di Milano-Bicocca\\
&  & *Universit\` a degli Studi ``G. d' Annunzio'' Chieti Pescara \\
\hline
12.&Latvia   & University of Latvia\\
&  & Transport and Telecommunication University \\
\hline
13.&Netherlands  &  Delft University of Technology\\
& & *Tilburg University \\   
& &  Utrecht University \\
 \hline
14.& Portugal  & Universidade Nova de Lisboa \\
 \hline
15.&Romania  & Babes-Bolyai Univ. Cluj-Napoca \\
\hline
16.&Spain  & *University of Almeria \\
&  & Universitat Politecnica de Catalunya \\
&  &  *University of Extremadura\\
&  &  *University Jaume I\\
&    & *University of M\' alaga \\
&  &  *Complutense University of Madrid\\
&  &  *University Oviedo\\
&  & *Universidad de Valladolid \\
\hline
17.& Sweden   &  Chalmers | Gothenburg University\\
\hline
18.&Switzerland   & University of Bern\\
& & EPFL \\
&  &  University of Lugano \\
 &    & ETH Z\" urich \\
&   &  University of Z\" urich\\
\hline
19.&UK   &  Cambridge University\\
&   &  University of Edinburgh\\
 &  & Imperial College London\\
&  & University of Oxford \\
\hline
\end{tabular}
%\vspace*{0.4cm}
\end{center}
%\caption{Participating Universities -- non IE members are marked with (*)}
\label{tab:names}
\end{table}
}}
\newpage
\section{Joint Degrees by Country}\label{apx:teaching}
\small{
\begin{table}[h]
\begin{center}
\label{tab:unis}
\begin{tabular}  {|l|l|l|}
\hline
{\bf Level}&{\bf Joint title} & {\bf Countries}\\
\hline
BSc & Economy and Computer Science & Spain, Switzerland \\
\hline
BSc & Economics and Business Informatics & Italy, Switzerland \\
\hline
BSc & Business Informatics  &  Austria, Czech, Germany \\
                                             &&Italy, Switzerland, UK, Denmark\\
\hline
BSc & Informatics and Management    & Italy,  UK \\
\hline
BSc & Informatics and Mathematics   & Netherlands, Spain, UK\\
\hline
BSc & Informatics and Statistics   & Spain\\
\hline
BSc & Informatics and Physics   & Spain, UK\\
\hline
BSc & Law and Informatics   &  Czech \\
\hline
BSc & Social sciences and Informatics   & Czech \\
\hline
BSc & Information Science /Library science   & Germany\\
\hline
BSc & Informatics Health  &Spain\\
\hline
BSc & Informatics and  Engineering   & Spain,  UK\\
\hline
BSc, MSc & Bioinformatics  & Czech, Denmark, Italy, Switzerland\\
\hline
BSc, MSc & Data Science   & Italy, Spain\\
\hline
BSc, MSc & Technical Communication    & Germany, Denmark\\
\hline
BSc, MSc & Computational Engineering   & Germany\\
\hline
MSc & ICT and Media   & Italy\\
\hline
MSc & Data Science and Entrepreneurship   & Netherlands\\
\hline
MSc & Data Science and Society   & Netherlands\\
\hline
MSc & Cognitive Science and Art. Intellig.   & Netherlands\\
\hline
MSc & Geoinformatics   & Italy\\
\hline
MSc & Data mining with political Sc.  &  Italy\\
\hline
MSc&  Informatics and Psychology   &  Italy\\
\hline
MSc & Cybernetics  &  Germany\\
\hline
MSc & Mechatronics   & Germany\\
\hline
MSc & INFOTech   & Germany\\
\hline
MSc & Comput. Sc. and Engineering & Switzerland\\
\hline
MSc& Bioinformatics  & Switzerland\\
\hline
MSc & Design Informatics    & UK, Denmark\\
\hline
\end{tabular}
%\vspace*{.4cm}
\end{center}
%\caption{Joint degrees (BSc and MSc) and countries}
\label{ tab:joint}
\end{table}
}


\end{document}  
